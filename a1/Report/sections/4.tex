\section{Analysis}

\subsection{Time complexity}
% This means the solution can be constructed by a chain of n/2 optimal solutions of sub problems.
In traditional divide and conquer where sub problems are divided by two, thus creating a runtime complexity $O(n\text{ } log\text{ } n)$, this algorithm is a little different.
Considering the context of the function $S(h,t)$ and the disregard of the otherwise and memory cases, then the following reoccurance applies.
$$ T(n) \leq \begin{cases}
    2T(n-2) + c & \text{if } n > 2 \\
    c & otherwise
\end{cases} $$ 

As combining the results, ie. taking max of the two solutions does not take $n$ time, only c is computed for each node considering a recursion tree. 
However, as the recursion is not dividing n by 2, the height is not $log \text{ } n$ but $\frac{n}{2}$.
Therefore we can infer the total amount of leaf nodes to be: $2^{\frac{n}{2}}$. Summing over all levels:

$$ T(n) \leq \sum_{i=0}^{\frac{n}{2}-1} 2^{i}c \leq c (2^{\frac{n}{2}} - 1) $$

With this, the runtime $O(2^{\frac{n}{2}})$ can be speculated, with asymptotic runtime $O(2^{n})$.
If we consider the otherwise case of $S(h,t)$, the runtime will be even worse than what is considered so far. 

The saving grace for the algorithm, is dynamic programming techniques that makes runtime bound by the substructure solution space which is of size $n^2$.
With the runtime bound by solution space, a better runtime complexity that we can infer is: $O(n^2)$ 

Although the worst case runtime complexity is $O(n^2)$ thanks to memoization, in practise it is significantly better.
The following diagram illustrates how the algorithm performs in practise given increasingly larger problems.

\textit{N.B. Card numbers are randomly generated between 1 and 500000}

DIAGRAM OF RUNS


\subsection{Space complexity}

Memory of solutions to subproblems will be stored in $M_{i,j}$.
If every sub problem is evaluated and stored, then the resulting space complexity is O$(n^2)$.
Worst case is very unlikely, and practically it is much less.

In the same manner as time complexity, the following diagram is of space complexity in practise.

DIAGRAM OF RUNS