\section{Solution}

\begin{itemize}
    \item Given $n$: total amount of cards as integer.
    \item Given $C_i$: as a list of $n$ integers, $C_1, C_2,..., C_n$.
    \item $h$ is defined as head index variable pointing to the leftmost card.
    \item $t$ is defined as tail index variable pointing to the rightmost card.
    \item $M_{i,j}$ is defined as solution to sub problem with the slice of $C$ where $i$ is the leftmost card and $j$ is the rightmost card.
\end{itemize}

Function to find best solution to sub problem where it is Donald's turn to pick a card is defined as follows:

$$ S(h,t) = \begin{cases}
    M_{h,t} & \text{if $M_{h,t}$ found} \\
    0 & \text{if $h = t$} \\
    \text{min}(C_h, C_t) & \text{if $|h-t| = 1$} \\
    \text{max} \begin{cases}
        C_{t-1} + S(h, t-2) \\
        C_h + S(h+1,t-1)
    \end{cases} & \text{if $C_h < C_t$ } \\
    \text{max} \begin{cases}
        C_t + S(h+1, t-1) \\
        C_{h+1} + S(h+2,t)
    \end{cases} & \text{if $C_h > C_t$ } \\
    \text{max} \begin{cases}
        C_{t-1} + S(h, t-2) \\
        C_h + S(h+1,t-1) \\
        C_t + S(h+1, t-1) \\
        C_{h+1} + S(h+2,t)
    \end{cases} & \text{otherwise} \\
\end{cases} $$

The functions captures the reuse of solutions of sub problems that have already been computed to avoid redudant computations, the 2 trivial bases cases, the case where leftmost card is highest, the case where rightmost card is highest and also if rightmost and leftmost are equal.

All succesfull evaluations will be memorized: $M_{h,t} = S(h,t)$, such that the solution is found for any subsequent computation of $S(h,t)$. 

The optimal solution of a given $C$ and $n$ is then found by the following:
$$ Opt_C = \begin{cases}
    \text{max}\begin{cases}
        C_1 + S(2,n) \\
        C_n + S(1,n-1)
    \end{cases} & \text{if $n > 1$} \\
    C_1 & \text{otherwise}
\end{cases} $$
